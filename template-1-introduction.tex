%% \section{Introduction} %for journal use above \firstsection{..} instead
Since the field of VA (Visual Analytics) was formally defined \cite{Fisher:2005} \cite{KOHLHAMMER2011117} the tools used to visualize data have evolved from desktop to a new breed of virtual reality based software and hardware apparatus which in some cases may facilitate the understanding by way of insights not readily available in 2 and 2.5D (3D rendered on a flat surface). There have been calls:
\begin{itemize}
\item to bring together Visual Analytics and Virtual Reality \footnote{Virtual Reality being understood as any combination of the
following characteristics: simulation, interaction,
artificiality, immersion, telepresence, full-body immersion,
and networked communications \cite{HEIM:1994}} aiming to understand how new interfaces and display technologies could be used to create immersive data analysis and exploration experiences \cite{Bach:2016:IAE:2992154.2996365} \cite{May:1667670}.

\item for the development of a research agenda TODO CLEANUP evolution of visualtization methods, aid perceptual limitations involved in grasping large amounts of data.\cite{Olshannikova2015}
\end{itemize}

Aligned with the mood we looked at developments in this crossover field, focusing on trajectory visual analytics, with findings presented below.